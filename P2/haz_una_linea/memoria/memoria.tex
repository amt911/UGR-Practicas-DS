\documentclass{article}
\usepackage[utf8]{inputenc}
\usepackage{graphicx, graphics, float}
\usepackage[a4paper, total={6in, 9in}]{geometry}
\title{Haz Una Línea\\
\large Memoria DS - Práctica 2}
\author{Andrés Merlo Trujillo\\ Sergio Hervás Cobo\\ Javier Serrano Lucas\\ Ricardo Molina Rodríguez}

\date{Abril 2022}
\begin{document}
\maketitle
\section{Introducción}
En esta práctica hemos decidido realizar con Flutter un videojuego semejante al conocido "Tetris".

\section{Descripción del sistema}
\subsection{Requisitos funcionales} 
\begin{itemize}
    \item El sistema debe permitir pausar la partida en cualquier
    momento.
    \item El sistema debe permitir al usuario manejar el movimiento de
    las piezas (traslación y rotación)
    \item El sistema debe permitir iniciar la partida.
\end{itemize}
\subsection{Requisitos no funcionales}
\begin{itemize}
    \item El sistema debe trabajar en tiempo real para evitar
    inconsistencias.
    \item El sistema tendrá una interfaz para el menú para elegir los
modos y comenzar la partida.
    \item Las físicas del sistema no deben estar unidas a los fotogramas
por segundo (FPS).
\end{itemize}
\subsection{Diagrama de algo no se de que xd}
\subsection{Diagrama de clases}
\end{document}
